\documentclass[a4paper]{jsarticle}
\usepackage{amsmath,amssymb}
\usepackage{bm}
\usepackage{color}
\usepackage{ascmac}

% global commands
\newcommand{\inv}{^{-1}}
\newcommand{\Gauss}[1]{\mathcal{N}(#1)}

% global macros
% define \vec[1] {\bm #1}
% define \mat[1] {\bm #1}

\begin{document}

\title{PRML3.3}
\author{サイボウズ・ラボ 西尾泰和}

% define \x \vec{x}
% define \y \vec{y}
% define \w \vec{w}
% define \t \vec{t}

% include BayesForGauss.tex


\section{PRML3.3.1}
% push
% define \m0 \vec{m_0}
% define \S0 \mat{S_0}

モデルパラメータ$\w$の事前分布を適当に決めてベイズ的なアプローチをしたい。
尤度関数が$\exp(-x^2)$の形なので、同じ形のガウス分布を事前分布にすれば形がキープされてうれしい。

そこで事前分布をガウス分布としよう。
\[ p(\w) = \Gauss{\w | \m0, \S0}\]

$\exp$の中だけに注目すると ($x \to y$は$x = a \exp ( {1 \over 2} y )$の意味で独自に使っている)
\[ p(\w) \to (\w - \m0)^T \S0\inv (\w - \m0) \]

尤度関数は
\[ p(\t | \w, \x) = \prod_n \Gauss{t_n | \w^T \phi(\x_n), \beta\inv} \]

$\exp$の中だけに注目すると
\[ p(\t | \w, \x) \to \beta \sum_n (t_n - \w^T \phi(\x_n))^2  \]

ベイズの定理より(念のため: 独立だから $p(\w) = p(\w | \x)$)
\[ p(\w | \t, \x) \propto p(\t | \x, \w) p(\w)\]

\begin{boxnote}
よって$p(\w | \t, \x)$の$\exp$の中身だけ考えると
\[ p(\w | \t, \x) \to \beta \sum_n \{ (t_n - \w^T \phi(\x_n))^2 \} + (\w - \m0)^T \S0\inv (\w - \m0)\]

展開して$\w$の二次の項、一次の項で整理してみよう。($\phi$ が $Phi$ に変わるところは前回やったけどやっぱわかりにくいな、今度書いておこう。)

\[ \beta\t^T\t -\beta\t^t Phi \w  -\beta Phi^T \w^T \t + \beta\w^T Phi^T Phi \w
+ 
\w^T \S0\inv\w
- \m0^T \S0\inv\w
- \w^T \S0\inv \m0
+ \m0^T \S0\inv \m0
\]

各項がスカラーなので転置しても良いことと$\S0$が対称行列なので転置しても同じ形であることを使って整理して
\[ \w^T(\S0\inv + \beta Phi^T Phi) \w - 2 \w^T (\S0\inv \m0 +\beta Phi^T \t)  + const\]

これで3.50, 3.51が示された
\end{boxnote}

このやり方の他にも、直接ガウス分布についてのベイズの定理を使う方法もある。

\BayesForGauss

\[ p(\w) = \Gauss{\w | \m0, \S0}\]
\[ p(\t | \w, \x) = \prod_n \Gauss{t_n | \w^T \phi(\x_n), \beta\inv} 
= \Gauss{\t | \Phi\w, \beta\inv \mat{I}}\]
\[ \x \to \w, \m \to \m0, \Lam \to \S0\inv, \y \to \t, \A = \Phi, \b \to \vec{0}, \L \to \beta \mat{I} \]
so
\eq3p49
where
\[ \Sn\inv = \S0\inv + \Phi^T \beta \mat{I} \Phi = \S0\inv + \beta \Phi^T \Phi \]
\[ \mn = \Sn(\Phi^T \beta \mat{I} \t + \S0\inv \m0) =  \Sn(\S0\inv \m0 + \beta \Phi^T \t)\]


% pop

\section{PRML3.3.2}
% push
% define \newt \hat{t}
% define \newx \vec{x}_{N+1}

実際問題として重みベクトルとか求めたいわけじゃなくて、新しい$\newx$が与えられたときにそれに対応する値$\newt$を推測したいってことの方が多い。だから$\w$を積分消去してしまおう。

っと思ったが実は積分消去しないでも$\w$を消す方法がある。p.90で行ったガウス変数に対するベイズの定理を使うのである。ベイズの定理を使えば$p(\x)$と$p(\y|\x)$から$p(\y)$を求める、つまり$\x$を消すことができる。

まずガウス分布についてのベイズの定理を思い出す。
\BayesForGauss

% define \xs \{\vec{x}_1,\ldots,\vec{x}_N\}
% define \mn \vec{m}_N
% define \Sn \mat{S}_N
% define \phix \vec{\phi}(\newx)

今回のケースで$\alpha, \beta, \xs, \t (= t_1, \ldots, t_N)$をまとめて$\Theta$と書くことにすれば、やりたいことは
$ p(\w | \Theta) $ と $ p(\newt | \w, \newx, \Theta) $ から $ p(\newt | \newx, \Theta) $ を求めることである。

PRML eq.3.49より
\[ p(\w | \Theta) = \Gauss{ \w | \mn, \Sn\inv } \]
PRML eq.3.8より
\[ p(\newt | \w, \newx, \Theta) = \Gauss{\newt | y(\newx, \w), \beta\inv } \]

ガウス分布におけるベイズの定理と見比べると、$ y(\newx, \w) = \w^T \phix$ で、これはスカラーだから転置していいので  $\A = \phix$。$\b = \vec{0}, \Lam = \beta, \L = \Sn \inv$となる。

よって

\[ p(\newt | \newx, \Theta) = \Gauss{\newt | \mn^T \phix, \beta\inv + \phix^T \Sn \phix}\]

これでPRML eq.3.58, eq.3.59が得られた。
% pop


\section{ex. 3.11}
% define SN1 S_{N+1}
% define SN S_N
% define xn1 x_{n+1}
$\forall x . x^T SN1 x \le x^T SN x$を示したい。
ex.3.8より$SN1\inv = SN\inv + \beta\phi(xn1)\phi(xn1)^T$
\[ SN1 = (SN\inv + \beta\phi(xn1)\phi(xn1)^T) \inv \]

\begin{itembox}[l]{Woodbury Identity}
\[ (A + B D\inv C)\inv = A\inv - A\inv B (D\inv + C A\inv B)\inv C A\inv \]
\end{itembox}

$A \to SN\inv, B \to \phi(xn1), C \to \phi(xn1)^T, D \to \beta\inv$なので
% push
% define P \phi(xn1)
% define Pt \phi(xn1)^T
\[ SN1 = SN - SN P (\beta\inv + Pt SN P)\inv Pt SN \]
$\beta$は正であり、$SN$が正定値であるから$Pt SN P$も正である。$a = (\beta\inv + Pt SN P)\inv$とおこう。
\[ x^T SN1 x = x^T SN x - a x^T SN P Pt SN x\]
\[ x^T SN x - x^T SN1 x = a x^T SN P Pt SN x\]
$SN$ は対称なので
\[ = a x^T (SN^T P Pt SN) x = a x^T (SN P) (SN P)^T x \]
$v = SN P$ とおこう
\begin{itembox}[l]{$vv^T$は半正定値}
任意のベクトル$v$について$vv^T$は半正定値である。なぜなら任意のベクトル$x$について
\[ x^T (vv^T) x = (x^Tv)(v^Tx) = (v^Tx)^T(v^Tx) = ||v^Tx||^2  \ge 0\]

\end{itembox}

よって与式は0以上である。

\end{document}

%\maketitle
