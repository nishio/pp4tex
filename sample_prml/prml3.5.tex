\documentclass[a4paper]{jsarticle}
\usepackage{amsmath,amssymb}
\usepackage{bm}
\usepackage{color}
\usepackage{ascmac}

% global commands
\newcommand{\inv}{^{-1}}
\newcommand{\Gauss}[1]{\mathcal{N}(#1)}

% global macros
% define \vec[1] {\bm #1}
% define \mat[1] {\bm #1}

\begin{document}
\if0

\title{PRML3.3}
\author{サイボウズ・ラボ 西尾泰和}
\maketitle
\section{3.5 エビデンス近似}
超パラメータ$\alpha, \beta$を人間が適当に決めるのとか気持ち悪いので、それもベイズ的に予測しよう。

% define \eq3p49:
\begin{itembox}[l]{eq 3.49}
% define \mn \vec{m_N}
% define \Sn \mat{S_N}
% define \w \vec{w}
% define \x \vec{x}
% define \t \vec{t}
\[ p(\w | \t, \x) = \Gauss{\t | \mn, \Sn}\]
\end{itembox}
% end

% define \eq3p53:
% define Phi \mat{\Phi}
\begin{itembox}[l]{eq 3.53, 3.54}
\[ \mn = \beta \Sn \Phi^T \t \]
\[ \Sn\inv = \alpha \mat{I} + \beta \Phi^T \Phi \]
\end{itembox}
% end

\eq3p49
\eq3p53

% define \a \alpha
% define \b \beta
% define \hata \hat{\a}
% define \hatb \hat{\b}

事後分布$p(\a, \b | \t)$が$\hata, \hatb$の周りでとても尖っている分布だと仮定するならば、
% define \newt \hat{t}
% define \newx \vec{x}_{N+1}
予測分布$p(\newt|\t)$を$\a, \b$がそれそれ$\hata, \hatb$だと固定して計算してしまっても、だいたいあってる。つまり
\[ p(\newt|\t) \simeq p(\newt | \t, \hata, \hatb) = \int p(\newt | \w, \hatb)p(\w | \t, \hata, \hatb) \]

$\hata, \hatb$はどうやって求める?ベイズの定理より$\a, \b$の事後分布$p(\a, \b | \t) \propto p(\t | \a, \b) p(\a, \b)$。
事前分布$p(\a, \b)$がとても平坦で定数とみなしていいんだと仮定するならば、$\hata, \hatb$は周辺尤度関数$p(\t | \a, \b)$を最大化することで求められる。

(sec. 3.5.1) 周辺尤度関数$p(\t | \a, \b)$はどうやって求める?同時分布を$\w$について積分することで得られる。

 \[ p(\t | \a, \b) = \myint{ p(\t | \w, b) p(\w | \a) }{w} \]

この積分を計算せずに、ガウス分布についてのベイズの定理を使って導くこともできるが、それはあとで演習3.16でやるのでここでは真面目に計算する。

% define \0 \vec{0}
% define \I \mat{I}
% define \X \mat{X}
\fi
PRMLの$p(\t | \w, \b)$は$p(\t | \w, \b, \X)$の$\X$が省略されていることに注意。

\begin{itembox}[l]{eq. 3.10(p. 138)}
\[ p(\t | \X, \w, \b) = \prod_n \Gauss{t_n | \w^T\phi(\x_n), \b\inv } \]
\end{itembox}

\begin{itembox}[l]{eq. 3.52(p. 152)}
\[ p(\w | \a) = \Gauss{\w | \0, \a\inv\I }\]
\end{itembox}

 \[ p(\t | \a, \b) = \myint{\prod_n \Gauss{t_n | \w^T\phi(\x_n), \b\inv }\Gauss{\w | \0, \a\inv\I}}{w} \]

\begin{itembox}[l]{$\exp$の中身に注目}
% push
% define \m \vec{\mu}
% define \S \vec{\Sigma}
$f(x) = y$を$x = \exp{-{1\over2}y}$の意味とすると:
\[ f(\prod_n x_n) \to \sum_n f(g_n)\]
\[ $f(\Gauss{\x | \m, \S\inv}) = (\x - \m)^T\S(\x - \m) \]
% pop
\end{itembox}

\[ f(\ldots) = \sum_n \b (t_n - \w^T\phi(\x_n))^2  + \a \w^T\w \]
\end{document}

